%this is the main document
\documentclass[11pt]{article}


% \usepackage{fullpage}

\usepackage{amssymb}

\usepackage{amsmath}

\usepackage{mathtools}

\usepackage{mathrsfs}

\usepackage{stmaryrd}

\usepackage{mathdots}

\usepackage{xcolor}

\usepackage{tikz-cd}

% \usepackage{sagetex}

\usepackage{pgfplots}

\usepackage{graphicx,wrapfig,lipsum}

\usepackage{selectp}

\usepackage{hyperref}

\usepackage{dsfont}

\usepackage[T1]{fontenc}

\usepackage[french]{babel}

\usepackage[utf8]{inputenc}
\usepackage[greek, french]{babel}
\usepackage{microtype}
\usepackage{amsmath,amsthm,amsfonts,amssymb}
\usepackage{hyperref}
\usepackage{graphicx}
\usepackage{lmodern}
\usepackage{amsmath,amsfonts,mathrsfs}
\usepackage{gensymb}
\usepackage{lettrine}
\usepackage{calligra}
\usepackage{tikz}
\usepackage{tikz-cd}
\usepackage{tikz-3dplot}
\usepackage{cancel}
\usepackage{boxedminipage}
\usepackage{mathtools}
\usepackage{esint}
\usepackage{stmaryrd}
\usepackage{minted}
\usepackage[european resistor, european voltage, european current]{circuitikz}
\usepackage{bussproofs}
\usepackage{yfonts}
\usepackage{turnstile}
\usepackage{dsfont}
\usepackage[
backend=biber,
style=alphabetic,
sorting=ynt
]{biblatex}
\addbibresource{biblio.bib}
\usepackage[
top=2cm, bottom=2cm, left=2cm , right=2cm]{geometry}

\makeatletter
\newdimen\@tempdimd
\makeatother
\usepackage{longfbox}

\usetikzlibrary{calc,patterns,angles,quotes}

\definecolor{amethyst}{rgb}{0.6, 0.4, 0.8}
\definecolor{brightturquoise}{rgb}{0.03, 0.91, 0.87}
\definecolor{mediumturquoise}{rgb}{0.28, 0.82, 0.8}
\definecolor{midnightblue}{rgb}{0.1, 0.1, 0.44}
\definecolor{gold}{rgb}{1.0, 0.84, 0.0}

\usetikzlibrary{arrows,shapes,positioning}
\usetikzlibrary{decorations.markings,decorations.pathmorphing,decorations.pathreplacing}
\usetikzlibrary{calc,patterns,shapes.geometric}

\usetikzlibrary{calc,patterns,angles,quotes}

\setminted[python]{breaklines, framesep=2mm, numbersep=5pt}



\pgfplotsset{compat=1.15}


\newtheorem{defn}{Définition}[subsubsection]

\newtheorem{propn}{Proposition}[subsubsection]

\newtheorem{coron}{Corollaire}[subsubsection]

\newtheorem{lemn}{Lemme}[subsubsection]




%\outputonly{1-17}

%\pagenumbering{gobble}



\begin{document}


%fonts
\newcommand{\bb}[1]{\mathbb{#1}}
\newcommand{\frk}[1]{\mathfrak{#1}}
\renewcommand{\cal}[1]{\mathcal{#1}}
\newcommand{\scr}[1]{\mathscr{#1}}


%problem
\newcommand{\problem}[1]{\vspace{3ex} {\centering
\fbox{ \begin{minipage}{1\textwidth} \vspace{2ex} \textbf{Problème. } #1 \vspace{2ex} \end{minipage} } \par} \vspace{3ex}  }
\newcommand{\sproblem}[1]{\vspace{3ex} {\centering
\fbox{ \begin{minipage}{1\textwidth} \vspace{2ex} \textbf{Suite du problème. } #1 \vspace{2ex} \end{minipage} } \par} \vspace{3ex}  }

%shortening
\renewcommand{\mod}[1]{\: (\mathrm{mod} \; #1)}
\newcommand{\floor}[1]{\left\lfloor #1 \right\rfloor}



%rm functions
\renewcommand{\Im}{\mathrm{Im}}
\newcommand{\Ker}{\mathrm{Ker}}
\newcommand{\mat}{\mathrm{mat}}
\newcommand{\rg}{\mathrm{rg}}
\newcommand{\tr}{\mathrm{tr}}
\newcommand{\Inv}{\mathrm{Inv}}
\newcommand{\card}{\mathrm{card}}
\newcommand{\id}{\mathrm{id}}
\newcommand{\Hom}{\mathrm{Hom}}
\newcommand{\End}{\mathrm{End}}
\newcommand{\Iso}{\mathrm{Iso}}
\newcommand{\Aut}{\mathrm{Aut}}

\renewcommand{\d}{\mathrm{d}}


%qed
\newcommand{\qed}{ \textcolor{white}{.}$\square$}


\tableofcontents

\newpage

\paragraph{Plan (provisoire)}
\begin{enumerate}
    \item \textbf{Une introduction :} Motivation du sujet (problématique écologique), rappel du travail précédent sur lequel on est censés rajouter, des simulations liées au modèle des fines couches
    \item \textbf{Cadre :} synthèse du problème 20, introduction du modèle à graphes, pertinence physique 
    \item \textbf{Capacité thermique apparente} (le filtre passe-bas à faire rentrer ici ?)
    \item Autres points à élaborer : 
    \begin{itemize}
        \item Graphe de diffusion comme circuit RC, grandeurs caractéristiques, temps carcatéristique : exemple avec deux corps, généralisation ? (tableau des temps caractéristiques 2 à 2)
        \item Diffusivité et effusivité
    \end{itemize}
\end{enumerate}

\newpage


%-------introduction
\section*{Introduction}
\addcontentsline{toc}{section}{Introduction} 



\newpage




%-------fin introduction

%-------section 1

\section{Cadre} %on pourra changer section en sous-section

\newpage 

%--------fin section 1

%-------section 2

\section{Capacité thermique apparente}

\newpage


%-------fin section 2






\end{document}


%\begin{center}

% \includegraphics[width=5cm]{circle approx.png}

%\end{center}